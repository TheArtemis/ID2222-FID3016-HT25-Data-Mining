% Very simple template for lab reports. Most common packages are already included.
\documentclass[a4paper, 11pt]{article}
\usepackage[utf8]{inputenc} % Change according to your file encoding
\usepackage{graphicx}
\usepackage{url}
\usepackage{listings}
\usepackage{xcolor}
\usepackage{float} % provides [H] placement to pin floats
\usepackage{placeins} % provides \FloatBarrier to prevent floats from passing barriers


%opening
\title{Report 4: Graph Spectra}
\author{Group 150: Lorenzo Deflorian, Riccardo Fragale}
\date{\today{}}

\begin{document}

\maketitle


\section{Introduction}

The goal of this homework is to implement the spectral graph clustering algorithm described in this paper~\url{https://ai.stanford.edu/~ang/papers/nips01-spectral.pdf} and use it to analyze two given graphs.

\section{Implementation}

The implementation is straightforward and can be found in the \texttt{miner/core/spectra/cluster_machine.py} module.

Given the graph represented by it's adjacency matrix, the steps to follow are:

\begin{enumerate}
    \item Remove loops from the graph by subtracting the diagonal of the graph from itself
    \item Build the degree matrix $D$
    \item Build the Laplacian matrix $L = D - A$
    \item Compute the eigenvalues and eigenvectors of the Laplacian matrix $L$
    \item We will take the first $k$ eigenvectors corresponding to the $k$ larges eigenvalues to build the feature matrix $X$

    \item We will then form the normalized feature matrix $Y$ to have unit length vectors for each row.
    \item We will then cluster the data using the k-means algorithm.
\end{enumerate}

The degree matrix $D$ is a diagonal matrix where the element $(i, i)$ is the degree of the $i$-th node, hence it just tells us how many edges are connected to the $i$-th node.

The Laplacian matrix $L$ is a symmetric matrix that is used to compute the eigenvalues and eigenvectors of the graph. It is defined as $L = D - A$, where $A$ is the adjacency matrix of the graph.

The idea behind the Laplacian matrix is to measure how far each node is from the other nodes. (... expand on this)

Once we have the feature matrix $X$, we can form the normalized feature matrix $Y$ to have unit length vectors for each row.

The k-means algorithm is used to cluster the data into $k$ clusters.

\section{Results}


\section{Conclusion}

\end{document}
