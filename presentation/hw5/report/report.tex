% Very simple template for lab reports. Most common packages are already included.
\documentclass[a4paper, 11pt]{article}
\usepackage[utf8]{inputenc} % Change according to your file encoding
\usepackage{graphicx}
\usepackage{url}
\usepackage{listings}
\usepackage[table]{xcolor}
\usepackage{float} % provides [H] placement to pin floats
\usepackage{placeins} % provides \FloatBarrier to prevent floats from passing barriers
\usepackage{booktabs} % for professional-looking tables
\usepackage{titlesec} % for customizing section headings

% Customize section spacing for better visual separation
\titlespacing*{\section}{0pt}{12pt plus 4pt minus 2pt}{8pt plus 2pt minus 2pt}
\titlespacing*{\subsection}{0pt}{10pt plus 4pt minus 2pt}{6pt plus 2pt minus 2pt}
\titlespacing*{\subsubsection}{0pt}{8pt plus 4pt minus 2pt}{4pt plus 2pt minus 2pt}

%opening
\title{Report 4: k-way graph partitioning using JaBeJa}
\author{Group 150: Lorenzo Deflorian, Riccardo Fragale}
\date{\today{}}

\begin{document}

\maketitle


\section{Introduction}
This assignment aims to understand distributed graph partitioning using gossip-based peer-to-peer techniques;
focusing in particular on the strategy suggested in the following paper\footnote{\url{https://publicatio.bibl.u-szeged.hu/5295/1/taas15.pdf}}.
We were given the skeleton of the algorithm and we needed to implement a couple of very important features
and then doing some tests and experimenting possible improvements.

\section{Task 1}
This first task was related to the initial implementation of the JaBeJa class.
We were asked to complete the implementation of the methods \textbf{sampleAndSwap(...) } and 
\textbf{findPartner(...)}.
\section{Task 2}



\section{Task 3}



\section{Conclusion}


\end{document}
